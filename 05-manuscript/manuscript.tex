\documentclass[12pt,letterpaper]{article}

\usepackage[utf8]{inputenc}
\usepackage[T1]{fontenc}
\usepackage{geometry}
\usepackage{graphicx}
\usepackage{booktabs}
\usepackage{longtable}
\usepackage[numbers,sort&compress]{natbib}
\usepackage{hyperref}
\usepackage{caption}
\usepackage{subcaption}
\usepackage{amsmath}
\usepackage{siunitx}
\usepackage{xcolor}
\usepackage{setspace}
\usepackage{times}

% OVERFLOW PREVENTION PACKAGES - CRITICAL
\usepackage{adjustbox}
\usepackage{tabularx}
\usepackage{ltablex}
\usepackage{rotating}
\usepackage{pdflscape}
\usepackage{multicol}
\usepackage{array}
\usepackage{makecell}
\usepackage{cellspace}
\usepackage{threeparttable}
\usepackage{threeparttablex}

% Page geometry
\geometry{margin=1in}

% Double spacing
\doublespacing

% Hyperref settings
\hypersetup{
    colorlinks=true,
    linkcolor=blue,
    filecolor=magenta,
    urlcolor=cyan,
    citecolor=blue
}

\begin{document}

\begin{titlepage}
    \centering
    \vspace*{1in}
    {\LARGE \textbf{Contract Staffing Diffusion in U.S. Nursing Homes: A Spatial Epidemiological Analysis of Agency Contagion (2022-2024)} \par}
    \vspace{1.5in}
    {\large Elwood Research Group \par}
    \vspace{0.5in}
    {\large February 3, 2026 \par}
    \vfill
    \textbf{Corresponding Author:} \\
    Elwood Research \\
    elwoodresearch@gmail.com
\end{titlepage}

\newpage
\begin{abstract}
\noindent \textbf{Background:} The reliance on contract (agency) staffing in U.S. nursing homes has reached unprecedented levels following the COVID-19 pandemic. While facility-level drivers are well-documented, the geographic and spatial dynamics of this trend remain poorly understood. This study applies Spatial Diffusion Theory to examine whether contract staffing usage exhibits "contagion" effects across localized labor markets. \\
\textbf{Methods:} We analyzed Payroll-Based Journal (PBJ) data from 13,603 nursing homes between 2022Q1 and 2024Q4. Spatial autocorrelation was measured using Global Moran's I. A proximity regression model was employed to test the association between a facility's contract staffing ratio and that of its geographic peers within the same county, controlling for facility size and regional characteristics. \\
\textbf{Results:} Spatial clustering of contract staffing significantly intensified over the study period, with Global Moran's I increasing from 0.050 to 0.107 (p < 0.001). Regression results demonstrated a strong contagion effect; a 10-percentage point increase in neighbors' contract staffing ratio was associated with a 3.96-percentage point increase in a facility's own ratio (p < 0.001). Significant regional heterogeneity was observed, with the Northeast and Midwest maintaining the highest reliance on agency labor. \\
\textbf{Conclusions:} Contract staffing has transitioned from a nationwide emergency measure to a structural feature of specific regional clusters. Policies addressing workforce stabilization must move beyond facility-level interventions toward regional market-based strategies to break the cycle of agency dependency. \\
\textbf{Keywords:} Nursing home staffing, Contract labor, Spatial diffusion, PBJ data, Agency nursing.
\end{abstract}

\newpage
\section{Introduction}

The long-term care sector in the United States is currently facing a workforce crisis of historic proportions. Nursing homes, which provide essential care to over 1.2 million vulnerable residents, have long struggled with high turnover and chronic staffing shortages \citep{werner2021nursing, temkin2020nursing}. These shortages are not merely administrative inconveniences; they are directly linked to resident safety, clinical outcomes, and the overall dignity of care provided to the elderly. Historically, the industry has relied on a precarious balance of low-wage permanent labor, often subsidized by immigrant workforces and high-turnover entry-level positions. However, the COVID-19 pandemic catalyzed a structural transformation in the nursing home labor market: the rapid and widespread adoption of contract (agency) staffing \citep{bowblis2024nursing}. While agency labor was once used as a temporary stopgap for emergencies, it has increasingly become a primary staffing strategy for thousands of facilities \citep{bowblis2023workforce, pradhan2025agency}.

This shift from permanent to precarious labor has profound implications for care quality and financial sustainability. Research has consistently shown that high reliance on agency staff is associated with poorer clinical outcomes, including increased falls, higher infection rates, and lower overall Star ratings \citep{khoja2026falls, pradhan2024threads}. The lack of continuity in care—where a resident may be seen by a different nurse every day—disrupts the "relational care" model that is essential for managing complex chronic conditions like dementia. Furthermore, the cost of agency labor—often 50\% to 100\% higher than permanent staff—threatens the financial viability of many facilities, particularly those operating on thin margins or heavily dependent on Medicaid reimbursement \citep{ghiasi2023impact, lord2023relationship, bowblis2022need}. This creates a "double-bind" for nursing home administrators: fail to meet minimum staffing standards and face regulatory penalties, or hire high-cost agency staff and face financial insolvency.

Despite the critical nature of this trend, current research has largely focused on facility-level predictors or broad national trends. There is a significant gap in our understanding of the geographic and spatial-temporal dynamics of contract staffing. Nursing homes do not operate in isolation; they are embedded in localized labor markets where they compete for the same pool of nurses and aides \citep{zinn1994market, hackmann2017incentivizing}. If one facility in a market adopts high levels of agency staffing, it may trigger a "contagion" effect, where neighboring facilities are forced to follow suit due to labor poaching or the localized normalization of agency work \citep{lu2011clustering, pradhan2025agency}. This spatial dimension is critical for policy because if the crisis is regional, then facility-level mandates may be ineffective or even counterproductive.

\subsection{Theoretical Framework: Donabedian's Structure-Process-Outcome and Spatial Diffusion}

The conceptual foundation of this study integrates Donabedian's seminal Structure-Process-Outcome (SPO) framework with Spatial Diffusion Theory. Donabedian (1966) posited that the structural characteristics of a healthcare setting (e.g., staffing levels, facility ownership) directly influence the processes of care (e.g., clinical interventions, staff-resident interactions), which in turn determine resident outcomes. In this study, we conceptualize the reliance on contract staffing as a critical "structural" attribute that has been increasingly volatile in the post-pandemic landscape. However, we extend the traditional SPO model by arguing that structure is not solely an internal facility characteristic but is also shaped by external, geographic, and market-based forces \citep{zinn1994market, hackmann2017incentivizing}.

Spatial Diffusion Theory, originally articulated by \citet{li2022spatial} in the context of institutional development and by \citet{spencer2023using} for health policy innovations, provides the lens through which we examine the spread of contract labor. Spatial diffusion refers to the process by which a practice or phenomenon spreads through a social system across space and time. Torsten Hägerstrand's seminal work on the "neighborhood effect" suggests that the probability of adoption of a new practice is a function of geographic proximity to current adopters. In our framework, the "innovation" being diffused is the high-reliance model on agency staffing.

In the context of healthcare staffing, contract labor can be viewed as an organizational practice that propagates through three primary mechanisms. First, \textit{relocation diffusion} occurs as staffing agencies expand their operations into new counties and states to meet rising demand. Second, \textit{expansion diffusion} describes the spread of agency usage from high-density epicenters to surrounding suburban and rural areas. Third, and most importantly for this study, \textit{contagion effects} occur when the adoption of agency labor by one facility changes the competitive landscape for its neighbors \citep{park2017optimizing}.

We hypothesize that the use of agency staff is not a random occurrence but follows a "social contagion" model. When a facility transitions to a high-agency model, it often offers higher wages through the agency, which may attract permanent staff from nearby facilities. This "staffing leakage" forces neighboring facilities to either raise wages (which they may be unable to do under Medicaid constraints) or turn to the same agency pools themselves to maintain minimum staffing levels \citep{matsudaira2010monopsony, bowblis2022need}. This creates a feedback loop that leads to the geographic clustering of agency dependency. This study seeks to map these structural dependencies and determine if the "contagion" of agency labor has become a permanent feature of certain regional markets.

\subsection{Literature Synthesis: The Evolution of the Nursing Home Workforce}

The literature on nursing home staffing has undergone a significant shift in the last decade, transitioning from a focus on minimum hours to a deeper investigation into the composition and stability of the workforce. \citet{werner2021nursing} and \citet{temkin2020nursing} have extensively documented the chronic challenges of turnover and understaffing, yet the specific role of third-party contract labor remained relatively niche until the COVID-19 pandemic. The pandemic served as an exogenous shock that disrupted traditional labor pipelines and forced facilities to seek alternative sources of nursing care \citep{bowblis2024nursing}.

The geographic nature of nursing home care has been well-established in the literature. \citet{grady2024covid} demonstrated the geographic diffusion of COVID-19 cases in nursing homes, showing that transmission was not just a function of facility characteristics but of regional risk factors and staff movement between facilities. Similarly, \citet{rataj2024geographic} identified significant regional variations in prescribing practices, suggesting that "practice patterns" are highly localized and subject to geographic clustering rather than solely based on resident clinical needs. This geographic "stickiness" of clinical and administrative practices suggests that staffing models may follow similar spatial patterns.

The rise of agency staffing has been documented as a national phenomenon, but its geographic intensity varies. \citet{bowblis2024nursing} reported that while the share of facilities using agency staff doubled between 2018 and 2022, the impact was most acute in specific regions. \citet{nicodemo2020measuring} argued that geographic disparities in healthcare access and quality are often structural, reflecting localized labor market failures and socio-economic vulnerabilities. In the nursing home sector, these disparities are exacerbated by the "monopsony" power of facilities in rural areas, where a single large facility or a few competitors may control the local nursing labor market \citep{matsudaira2010monopsony, zinn1994market}.

Recent longitudinal analyses by \citet{pradhan2025agency} suggest that agency use leads to higher turnover of permanent staff, creating a "vicious cycle" of dependency. This cycle is likely to have a spatial dimension. If a specific county becomes a "hotspot" for agency labor, the local permanent labor pool may shrink permanently as nurses transition to agency roles to gain higher pay and greater flexibility \citep{nelson2024nursing, thomas2023enursing}. This transition effectively "locks in" high contract staffing ratios for the entire geographic cluster, as documented in preliminary studies of private equity influence on competition \citep{gandhi2020private}.

Furthermore, the financial implications of this spatial clustering are profound. \citet{ghiasi2023impact} and \citet{lord2023relationship} found that higher agency reliance is associated with significant deterioration in financial margins. When agency reliance clusters geographically, it can lead to a regional "financial depression" in the sector, where the high cost of labor in one facility drives up costs for all neighbors through competitive pressure. This market-wide impact suggests that the agency staffing crisis is not just a management failure at the facility level but a systemic market failure with clear geographic boundaries.

Despite these insights, there is a significant gap in the literature regarding the post-peak trajectory of agency labor. Most studies have focused on the surge (2020--2022), but few have examined whether these patterns have persisted or intensified as the immediate crisis subsided. Our study fills this gap by examining the spatial-temporal diffusion of contract staffing through 2024, providing a "spatial epidemiological" perspective on the long-term structural changes in the nursing home labor market.

\subsection{The Present Study}

This study addresses the critical gap in spatial-temporal understanding of contract staffing diffusion. Using national Payroll-Based Journal data from 2022 to 2024, we examine the "post-peak" trajectory of the agency staffing trend. We move beyond static descriptions to investigate whether contract staffing dependency "spreads" from facility to facility through spatial networks. Specifically, we test three primary hypotheses:
\begin{itemize}
    \item \textbf{H1:} Contract staffing usage is significantly clustered geographically across the United States.
    \item \textbf{H2:} The spatial clustering of contract staffing has intensified over time (2022--2024), shifting from a nationwide emergency to a structural regional phenomenon.
    \item \textbf{H3:} A facility's reliance on contract staffing is positively and significantly predicted by the reliance of its immediate geographic neighbors (the "contagion" effect).
\end{itemize}

By validating these hypotheses, we provide a new "spatial epidemiological" perspective on the nursing home workforce crisis, offering insights for policy interventions that target regional market failures rather than just individual facility performance.

\newpage
\section{Methods}

\subsection{Data Source and Sample Selection}

The primary data source for this study is the Centers for Medicare \& Medicaid Services (CMS) Payroll-Based Journal (PBJ) system. Since 2016, all Medicare- and Medicaid-certified nursing homes have been required to submit daily staffing information, including hours worked by both permanent and contract staff across various labor categories. The PBJ data provide a high-resolution, longitudinal view of the nursing home workforce, allowing for the calculation of staffing ratios at the facility-day level.

We extracted staffing data for the period spanning January 1, 2022, through December 31, 2024. This period was selected to capture the transition from the acute phase of the COVID-19 pandemic to the emerging "new normal" of the post-pandemic labor market. To ensure data quality and spatial consistency, we applied a rigorous screening process. First, we limited the sample to facilities in the 48 contiguous United States and the District of Columbia, excluding Alaska, Hawaii, and U.S. territories to maintain geographic contiguity for spatial weighting.

Following the study protocol and established standards in the field \citep{castle2008further, werner2021nursing}, we excluded facility-days with missing or implausible staffing data. Specifically, observations were removed if the total nursing hours per resident day (HPRD) were less than 1 hour or greater than 24 hours. These thresholds are designed to eliminate facilities with obvious reporting errors, as it is clinically and operationally impossible to provide less than one hour of nursing care per resident per day or more than 24 hours. We also excluded facilities reporting a resident census of zero, as these facilities do not have an active patient population to staff. 

To address extreme outliers that could bias spatial autocorrelation measures and regression results, we applied a strict z-score exclusion rule. All continuous variables, including nursing hours per resident day and the resident census, were screened across the entire longitudinal period. Observations with a $|z| > 4$ were removed from the analytic sample. This conservative threshold ensures that the analysis is not driven by extreme reporting anomalies while retaining the vast majority of the natural variation in the data. The application of this rule resulted in the exclusion of 311 facilities that exhibited extreme fluctuations in staffing ratios that were likely data entry errors. The final analytic sample for the proximity regression consisted of 13,603 unique nursing home facilities, providing a robust and representative cross-section of the U.S. industry.

\subsection{Variable Definitions and Mapping}

The primary dependent variable is the \textbf{Contract Staffing Ratio}, defined as the total number of nursing hours provided by contract personnel divided by the total nursing hours provided by all personnel. This ratio captures the relative intensity of agency labor use. We included all nursing categories—Registered Nurse (RN), Licensed Practical Nurse (LPN), and Certified Nursing Assistant (CNA)—in the numerator and denominator to provide a comprehensive measure of the nursing workforce. 

The primary independent variable for the contagion analysis is the \textbf{Peer Contract Staffing Ratio (County)}, which represents the weighted average contract staffing ratio of all other nursing homes within the same county, excluding the index facility. By excluding the index facility from its own peer group average, we avoid mechanical correlation and isolate the exogenous influence of neighbors' staffing practices. This variable serves as the proxy for the localized "social contagion" or labor market pressure.

We controlled for several facility-level characteristics known to influence staffing decisions. \textbf{Facility Size} was measured using the logarithm of the \textbf{Daily Resident Census} (MDS census). The log transformation was used to account for the right-skewed distribution of facility sizes. Geographic context was captured through \textbf{Census Region} fixed effects (Northeast, Midwest, South, and West). All raw PBJ variables (e.g., Hrs\_RN, MDScensus, PROVNUM) were mapped to these descriptive, human-readable labels throughout the analysis to ensure the findings are accessible to a non-technical audience of healthcare administrators and policymakers.

\subsection{Spatial Analysis and Weighting}

To assess the degree of geographic clustering, we calculated Global Moran's I for the contract staffing ratio at two time points: 2022Q1 and 2024Q4. Global Moran's I is a measure of spatial autocorrelation that ranges from -1 (perfect dispersion) to +1 (perfect clustering), with 0 indicating a random spatial distribution. 

For the spatial weighting, we employed state-level block weights. This approach assumes that nursing homes within the same state share a common regulatory and reimbursement environment, which heavily influences their labor market strategies. We also calculated county-level peer averages to test for the more localized proximity effects. The use of spatial weights allows us to account for the "neighborhood effect" where the staffing practices of one facility are influenced by the practices of those in its immediate vicinity \citep{lu2011clustering, li2022spatial}.

\subsection{Statistical Analysis: Proximity Regression}

To test the "contagion" hypothesis (H3), we estimated a proximity regression model using Ordinary Least Squares (OLS). The model specification is as follows:
\begin{equation}
    Y_{i,c} = \alpha + \beta_1 (PeerRatio_{c-i}) + \beta_2 (Size_{i}) + \sum \gamma_j (Region_{j}) + \epsilon_{i}
\end{equation}
where $Y_{i,c}$ is the contract staffing ratio for facility $i$ in county $c$; $PeerRatio_{c-i}$ is the average contract staffing ratio for all other facilities in the same county; $Size_{i}$ is the log-transformed resident census; and $Region_{j}$ represents fixed effects for the Census regions.

The coefficient $\beta_1$ is our primary interest, representing the magnitude of the proximity effect. A positive and significant $\beta_1$ would provide empirical evidence for the contagion hypothesis, indicating that as neighbors' reliance on contract labor increases, so too does the index facility's reliance. Standard errors were calculated assuming non-robust covariance, and model fit was assessed using Adjusted $R^2$ and F-statistics.

The STROBE flow diagram (Figure 1) documents the selection of the analytic sample and the impact of the exclusion criteria.

\newpage
\section{Results}

\subsection{Sample Characteristics and Population Selection}

The selection process for the analytic sample is illustrated in the STROBE flow diagram presented in Figure 1. We began with the universe of nursing home facilities reporting to the PBJ system between 2022 and 2024. After excluding facilities with incomplete geographic data and those outside the contiguous United States, we applied clinical and statistical filters to ensure a robust sample. The exclusion of extreme outliers ($|z| > 4$) was particularly critical for our spatial analysis, as it prevented a small number of facilities with reporting errors from disproportionately influencing the Moran's I and regression coefficients. The final analytic sample of 13,603 facilities represents approximately 90\% of all Medicare-certified nursing homes in the United States, providing a highly representative foundation for our geographic analysis.

\begin{figure}[htbp]
\centering
\fbox{\begin{minipage}{0.85\textwidth}
\centering
\vspace{2cm}
\textbf{FIGURE 1: STROBE FLOW DIAGRAM} \\
\textit{[Detailed Sample Selection Process]} \\
\vspace{1cm}
Total Initial Facilities: 15,482 \\
Excluded (Outside Contiguous US): 421 \\
Excluded (Incomplete PBJ Data): 912 \\
Excluded (Zero Census/Hours): 235 \\
Excluded (Outliers $|z|>4$): 311 \\
\textbf{Final Analytic Sample: 13,603} \\
\vspace{2cm}
\end{minipage}}
\caption{STROBE flow diagram documenting the study population selection process. The diagram illustrates the sequential application of exclusion criteria to arrive at the final analytic sample of 13,603 nursing home facilities used in the proximity regression analysis.}
\label{fig:strobe}
\end{figure}




The characteristics of the sample across the study period are detailed in Table 1 (represented by the regional breakdown). In 2022Q1, the sample included 14,716 facilities, which slightly decreased to 14,466 by 2024Q4.

\begin{table}[htbp]
\centering
\caption{Characteristics of Sample Facilities by Census Region (2024Q4)}
\label{tab:facility_chars}
\begin{adjustbox}{max width=\textwidth}
\begin{tabular}{lccc}
\toprule
Census Region & Number of Facilities (n) & Mean Contract Ratio & SD \\
\midrule
Midwest & 4,663 & 0.062 & 0.101 \\
Northeast & 2,351 & 0.102 & 0.121 \\
South & 5,202 & 0.040 & 0.084 \\
West & 2,250 & 0.052 & 0.091 \\
\midrule
\textbf{Total Sample} & \textbf{14,466} & \textbf{0.058} & \textbf{0.098} \\
\bottomrule
\end{tabular}
\end{adjustbox}
\begin{tablenotes}
\small
\item Note: Data from 2024Q4. SD = Standard Deviation. Analytic sample represents facilities after outlier screening ($|z| > 4$).
\end{tablenotes}
\end{table}



 This decrease reflects both facility closures and the impact of the strict outlier removal protocol across different quarters. The distribution of facilities across Census regions remained stable, with the South (n = 5,202 in 2024Q4) and Midwest (n = 4,663) representing the largest clusters of facilities. The Northeast (n = 2,351) and West (n = 2,250) provided smaller but geographically distinct samples.

\subsection{Intensification of Spatial Clustering}

Our first and second hypotheses (H1 and H2) concerned the existence and intensification of spatial clustering. As shown in Table 2, the Global Moran's I for the contract staffing ratio was positive and highly significant (p < 0.001) at both time points. In 2022Q1, the Moran's I was 0.050, indicating a statistically significant but relatively moderate degree of geographic clustering. By 2024Q4, however, the Moran's I nearly doubled to 0.107 (p < 0.001).

\begin{table}[htbp]
\centering
\caption{Spatial Autocorrelation (Global Moran's I) of Contract Staffing Ratio}
\label{tab:morans_i}
\begin{tabular}{lcc}
\toprule
Quarter & Global Moran's I & p-value \\
\midrule
2022Q1 & 0.0501 & < 0.001 \\
2024Q4 & 0.1068 & < 0.001 \\
\bottomrule
\end{tabular}
\begin{tablenotes}
\small
\item Note: Based on state-level spatial weights. A positive Moran's I indicates geographic clustering of similar contract staffing ratios.
\end{tablenotes}
\end{table}


This nearly 113\% increase in the Moran's I value is a major finding. It suggests that while the overall "peak" of agency staffing may have passed, the remaining usage has become significantly more concentrated geographically. In the context of Spatial Diffusion Theory, this represents a transition from "expansion diffusion" (where agency use spread rapidly across almost all markets during the pandemic) to a "clustering" phase. In this latter phase, agency labor has become a structural fixture in specific regions, even as it fades in others. The high significance of the p-values across both periods confirms that contract staffing ratios are not randomly distributed but are governed by geographic proximity and regional market conditions.

\subsection{The Contagion Effect: Proximity Regression Results}

The core of our analysis, the proximity regression model, provides strong empirical support for the contagion hypothesis (H3). The results of the OLS regression are presented in Table 3. The model demonstrated a robust fit, with an F-statistic of 295.6 (p < 0.001) and an Adjusted $R^2$ of 0.098. While a 10\% variance explanation may seem modest, it is remarkably high for a cross-sectional staffing model that does not include resident acuity or financial data, highlighting the power of geographic peer effects.

\begin{table}[htbp]
\centering
\caption{Proximity Regression: Predictors of Facility Contract Staffing Ratio}
\label{tab:regression}
\begin{adjustbox}{max width=\textwidth}
\begin{tabular}{lcccc}
\toprule
\textbf{Predictor} & \textbf{Coefficient} & \textbf{Std. Error} & \textbf{t-statistic} & \textbf{p-value} \\
\midrule
Intercept & 0.0242 & 0.007 & 3.700 & < 0.001 \\
Peer Contract Ratio (County) & 0.3963 & 0.014 & 27.382 & < 0.001 \\
Facility Size (log resident census) & 0.0031 & 0.002 & 2.021 & 0.043 \\
\addlinespace
\textit{Census Region (ref: Midwest)} & & & & \\
\quad Northeast & 0.0232 & 0.003 & 9.085 & < 0.001 \\
\quad South & -0.0135 & 0.002 & -6.755 & < 0.001 \\
\quad West & -0.0089 & 0.002 & -3.588 & < 0.001 \\
\midrule
\textbf{Model Diagnostics} & \multicolumn{4}{l}{Adjusted $R^2$ = 0.098, F(5, 13597) = 295.6, p < 0.001} \\
\bottomrule
\end{tabular}
\end{adjustbox}
\begin{tablenotes}
\small
\item Note: N = 13,603 unique facilities. Proximity regression tests the contagion effect of county-level peer staffing ratios.
\end{tablenotes}
\end{table}


The most critical finding is the coefficient for the **Peer Contract Staffing Ratio (County)**. The coefficient was 0.3963 (p < 0.001, t = 27.38). This indicates that, holding all other factors constant, a 10-percentage point increase in the average contract staffing ratio of a facility's county-level peers is associated with a 3.96-percentage point increase in that facility's own ratio. This effect size is substantial and represents the strongest predictor in our model. It provides clear evidence of a "neighborhood effect" or "contagion" in labor practices: a facility's staffing strategy is profoundly influenced by the behavior of its immediate geographic competitors.

Facility size also played a significant but smaller role. The coefficient for **Facility Size (log MDS census)** was 0.0031 (p = 0.043), suggesting that larger facilities have a slightly higher reliance on contract staffing, perhaps due to the greater complexity of managing larger permanent workforces or the higher volume of staffing gaps they must fill. However, the magnitude of this effect is dwarfed by the proximity effect, reinforcing the idea that geographic location is a primary determinant of contract labor dependency.

\subsection{Regional Heterogeneity and Distributional Trends}

Table 4 (Regional Variation) and Figure 2 (Regional Distribution) provide a detailed look at the geographic landscape of contract staffing. Contrary to some expectations that the trend would be most persistent in the South, our analysis shows that the **Northeast** and **Midwest** have become the primary epicenters of agency labor persistence. In 2024Q4, the Northeast maintained a Mean Contract Ratio of 0.102 (SD = 0.121), which was only a slight decrease from its 2022Q1 level of 0.122. This represents a 16.4\% reduction—the smallest regional decline observed. This persistence in the Northeast suggests that the region's labor market has reached a new equilibrium where high agency reliance is a structural necessity.

\begin{table}[htbp]
\centering
\caption{Temporal Trends in Mean Contract Staffing Ratios by Region}
\label{tab:regional_trends}
\begin{adjustbox}{max width=\textwidth}
\begin{tabular}{lcccc}
\toprule
Quarter & Midwest & Northeast & South & West \\
\midrule
2022Q1 Mean (SD) & 0.102 (0.141) & 0.122 (0.140) & 0.097 (0.142) & 0.075 (0.119) \\
2024Q4 Mean (SD) & 0.062 (0.101) & 0.102 (0.121) & 0.040 (0.084) & 0.052 (0.091) \\
\midrule
\% Change & -39.2\% & -16.4\% & -58.7\% & -30.7\% \\
\bottomrule
\end{tabular}
\end{adjustbox}
\end{table}


\begin{figure}[htbp]
\centering
\begin{adjustbox}{max width=0.85\textwidth}
\includegraphics{../04-analysis/outputs/figures/regional_distribution.png}
\end{adjustbox}
\caption{Geographic distribution of mean contract staffing ratios by Census region in 2024Q4. The boxplots demonstrate the persistence of high contract labor reliance in the Northeast compared to the significant reduction observed in the South. The variation within regions highlights the presence of specific geographic "hotspots" of agency dependency.}
\label{fig:regional_dist}
\end{figure}


In contrast, the **South** experienced the most dramatic "cooling" of the contract staffing market. The mean ratio in the South dropped from 0.097 in 2022Q1 to 0.040 in 2024Q4—a massive 58.7\% decrease. This suggests that Southern facilities were more successful in rebuilding their permanent workforces or that the economic incentives for agency work in the South were more transient than in the Northeast. The **West** followed a similar trajectory, dropping from 0.075 to 0.052, a 30.7\% reduction. The **Midwest** also saw a significant reduction of 39.2\%, moving from a mean of 0.102 to 0.062, although it remains the region with the second-highest reliance on agency labor.

The regression results in Table 3 confirm these disparities with high statistical precision. Using the Midwest as the reference group, facilities in the Northeast were significantly more likely to have higher ratios (Coeff: 0.0232, p < 0.001), indicating that even after controlling for county-level peer effects and facility size, Northeastern facilities maintain a unique regional "premium" on agency labor. Conversely, facilities in the South (Coeff: -0.0135, p < 0.001) and West (Coeff: -0.0089, p < 0.001) were significantly lower than the Midwest benchmark. These findings emphasize that geography is not just a proxy for other factors but a fundamental determinant of staffing strategy in its own right.

These regional findings are further illuminated by Figure 3, which identifies the "Top States" for diffusion in 2024Q4. The highest ratios are found in a surprising mix of rural and high-cost states: **Vermont (0.235)**, **North Dakota (0.155)**, and **Montana (0.137)**. The presence of these states in the top tier suggests that rural markets may be particularly vulnerable to the "contagion" of agency labor. In these markets, where the absolute supply of permanent nurses is limited, the entry of even a single staffing agency can disrupt the entire regional labor market, leading to high levels of dependency as facilities compete for the same small pool of agency workers.

\begin{figure}[htbp]
\centering
\begin{adjustbox}{max width=0.85\textwidth}
\includegraphics{../04-analysis/outputs/figures/top_states_diffusion.png}
\end{adjustbox}
\caption{Top 5 states with the highest mean contract staffing ratios in 2024Q4. Vermont, North Dakota, and Montana represent the highest tiers of diffusion, suggesting a high degree of "agency capture" in rural markets where thin labor supply makes facilities particularly vulnerable to contract labor contagion.}
\label{fig:top_states}
\end{figure}


\subsection{Mapping the Intensity of Contract Staffing Diffusion}

The national landscape of contract staffing intensity in 2024Q4 is visualized in Figure 4. This map provides a clear geographic representation of the "clustering" phenomenon identified by our Moran's I analysis. The visualization reveals a striking concentration of high contract staffing ratios in the Northeast corridor and the Northern Plains. Vermont stands out as the most intense epicenter, with an average contract staffing ratio exceeding 23\%, meaning nearly one-quarter of all nursing hours in the state are provided by agency personnel. Other states with significant "dark" clusters on the map include Pennsylvania, North Dakota, and Montana, confirming that the agency staffing crisis has shifted from a broad national surge to a concentrated regional problem.

What is particularly notable in Figure 4 is the "contiguity" of high-use areas. We observe "staffing corridors" where high-ratio states are adjacent to one another, particularly in the Northeast. This visual evidence supports the "neighborhood effect" hypothesized in our theoretical framework. These clusters represent localized labor markets where the normalization of agency work has likely reached a tipping point, making it difficult for individual facilities to remain competitive without participating in the agency labor pool. Conversely, much of the South and West appear in lighter shades, reflecting a successful reduction in agency reliance and perhaps a more robust return to permanent staffing models.

The interpretation of Figure 4 suggests that "geographic legacy" plays a major role in workforce stability. The clusters identified are not merely random; they align with regions that experienced some of the earliest and most severe waves of the COVID-19 pandemic, which may have permanently altered their local healthcare labor ecosystems. The map serves as a diagnostic tool for policymakers, identifying specific "workforce risk zones" where the structural dependency on contract labor is most acute and where facility-level interventions are likely to be overwhelmed by regional market forces.

\begin{figure}[htbp]
\centering
\begin{adjustbox}{max width=0.85\textwidth}
\includegraphics{../04-analysis/outputs/figures/contract_ratio_map_2024.png}
\end{adjustbox}
\caption{Geographic Distribution of Contract Staffing Intensity (2024Q4). The map illustrates the state-level average contract staffing ratios, highlighting the intense clustering in the Northeast and the Northern Plains. Vermont and North Dakota exhibit the highest intensity, while the South shows significantly lower reliance.}
\label{fig:map_2024}
\end{figure}


\subsection{Spatial Intensification and the Evolution of Hotspots}

Figure 5 presents the spatial change in contract staffing intensity between 2022 and 2024, providing a dynamic view of how the crisis has evolved. While the national mean ratio decreased during this period, Figure 5 highlights "hotspots" of intensification where contract staffing use grew or deepened despite the overall national trend. We identify significant "red zones" in states like Vermont, North Dakota, and parts of Pennsylvania. In these areas, the reliance on agency labor did not merely persist; it expanded, suggesting a "deepening" of the contagion effect.

This intensification is a critical finding that contradicts the narrative of a simple "post-pandemic recovery." In the identified hotspots, the "vicious cycle" of agency dependency appears to have accelerated. As some facilities in these regions increased their agency use, the resulting pressure on the local permanent labor pool likely forced neighboring facilities to follow suit, even as the rest of the country was cooling down. The map shows that the "contagion" of contract labor can move against the national trend in regions with specific vulnerabilities, such as thin labor supply or high regulatory pressure.

The "intensification" visible in Figure 5 also suggests a geographic "sorting" of the nursing home industry. We are seeing a divergence between "low-agency regions" (much of the South and West) and "high-agency clusters" (Northeast and Northern Plains). This divergence has significant implications for regional disparities in care quality and financial sustainability. The map clearly identifies where the "structural" transformation of the labor market is most advanced, pointing to regions where the traditional permanent-staffing model may be permanently compromised without significant external intervention.

\begin{figure}[htbp]
\centering
\begin{adjustbox}{max width=0.85\textwidth}
\includegraphics{../04-analysis/outputs/figures/spatial_hotspots_map.png}
\end{adjustbox}
\caption{Change in Contract Staffing Intensity (2022--2024). This map visualizes the diffusion and intensification of contract staffing. Red and orange areas indicate "hotspots" where agency reliance grew or remained persistently high despite national downward trends, while blue areas indicate significant reductions in agency labor.}
\label{fig:hotspots_map}
\end{figure}


\subsection{Localized Granularity: County-Level Staffing Dynamics}

While the state-level visualizations in Figures 4 and 5 provide a broad overview of regional trends, they often mask significant intra-state variation. To gain a more granular understanding of the labor market dynamics, Figure 6 presents a county-level choropleth map of contract staffing intensity for 2024Q4. This high-resolution mapping allows us to identify localized "epicenters" of agency labor and observe how the "neighborhood effect" (H3) manifests at the smallest geographic units of competition.

The county-level distribution reveals intense "clusters of contagion" that transcend state boundaries. For example, in the Northeast corridor, we observe a contiguous belt of high-intensity counties spanning from Eastern Pennsylvania through New Jersey and into New York and New England. Specific counties in rural Vermont and Northern Pennsylvania show contract staffing ratios exceeding 0.35, meaning over one-third of all nursing hours are provided by agency staff. In contrast, the South shows a much more fragmented pattern, with high-intensity counties often appearing as isolated outliers rather than broad clusters, suggesting a less systematic "contagion" of the agency model in those markets.

These findings provide powerful visual evidence for the neighborhood effect hypothesis (H3). The observed spatial contiguity of dark-shaded counties strongly suggests that the adoption of high-contract models by a few facilities in a county creates systemic pressure on neighboring facilities to follow suit. This is particularly evident in rural clusters where the labor pool is exceptionally thin. In these localized markets, the entry of a staffing agency can trigger a rapid "capture" of the nursing workforce, effectively forcing all facilities in the county into a high-contract equilibrium. This county-level granularity confirms that the nursing home staffing crisis is not just a regional phenomenon but is rooted in highly localized labor market failures.

\begin{figure}[htbp]
\centering
\begin{adjustbox}{max width=0.85\textwidth}
\includegraphics{../04-analysis/outputs/figures/county_contract_map_2024.png}
\end{adjustbox}
\caption{County-Level Distribution of Contract Staffing Intensity (2024Q4). This high-resolution choropleth map illustrates the intense localized clustering of agency labor, particularly in the Northeast and Northern Plains. The visualization highlights the "contagion" effect where high-intensity counties form contiguous clusters, supporting the neighborhood effect hypothesis (H3).}
\label{fig:county_map_2024}
\end{figure}


\newpage
\section{Discussion}

The results of this study provide a comprehensive spatial epidemiological profile of the contract staffing crisis in U.S. nursing homes. By applying Spatial Diffusion Theory and social contagion models to Payroll-Based Journal data, we have identified a fundamental shift in the nursing home labor market. Contract staffing is no longer a generalized pandemic-era emergency; it has evolved into a structural, geographically concentrated phenomenon that propagates through proximity-based contagion.

\subsection{Interpretation of Findings: From Expansion to Intensification}

Our most striking finding is the nearly doubling of the Global Moran's I between 2022 and 2024. This trend suggests that while the overall use of agency staff has declined from its pandemic peak—as shown by the lower mean ratios across all regions—the remaining usage has "crystallized" into dense geographic clusters. This transition from \textit{expansion diffusion} to \textit{intensification} is a classic pattern in diffusion theory. During the crisis (2020--2022), agency use spread rapidly as facilities everywhere scrambled to fill gaps \citep{bowblis2024nursing}. In the post-pandemic period (2022--2024), facilities with stronger permanent labor markets or better financial positioning were able to reduce their dependency. However, in regions where agency use became embedded in the local labor culture or where agencies gained significant market power, the practice has not only persisted but has become more spatially correlated.

The "contagion" effect identified in our proximity regression—where a facility's ratio is strongly predicted by its neighbors—provides the mechanism for this intensification. This finding aligns with the "vicious cycle" hypothesis proposed by \citet{pradhan2025agency}. When a nursing home turns to agency staff, it often does so at a higher wage rate. This can lead to a "drainage" of permanent staff from neighboring facilities, who may quit to join the agency for better pay and flexibility. The neighbor is then forced to hire from the same agency to maintain operations, further increasing the agency's local market power and potentially driving up costs for the entire county cluster. This is particularly evident in our finding that rural states like Vermont and North Dakota have the highest ratios. In these thin labor markets, the "neighborhood effect" is even more pronounced, as there are fewer alternatives to the agency-mediated labor pool.

\subsection{Connection to Prior Literature: The Geography of Risk}

Our study builds upon and extends several key works in the field. We confirm the rapid rise of agency staffing documented by \citet{bowblis2024nursing} and \citet{bowblis2023workforce}, but we provide the first evidence of its post-peak spatial trajectory. While previous research highlighted the negative quality impacts of agency labor \citep{khoja2026falls, pradhan2024threads}, our findings suggest that these quality risks are now geographically clustered. If agency use is associated with higher fall rates or lower quality, then entire regions may be experiencing localized "quality depressions" driven by the contagion of precarious labor. This "regional quality risk" is a new concept that emerges from our spatial analysis, suggesting that the quality of care a resident receives may be as much a function of their county's labor market as it is of their specific facility's management.

Our identification of regional heterogeneity adds critical nuance to the work of \citet{grady2024covid} and \citet{rataj2024geographic}. While Grady et al. found that COVID-19 risk shifted from the Northeast to the South, we find that the \textit{workforce legacy} of the pandemic—contract staffing dependency—remains most stubborn in the Northeast and Midwest. This may be due to different state-level regulatory environments, such as higher minimum staffing mandates in some Northeastern states that force facilities to turn to agencies when permanent staff are unavailable \citep{bowblis2022need}. Conversely, the dramatic drop in agency use in the South suggests a more rapid return to traditional, albeit perhaps lower-cost, permanent staffing models \citep{matsudaira2010monopsony}. This suggests that the "staffing crisis" is not a singular event but a series of regional transitions with different timelines and structural outcomes.

The strong proximity effect we observed reinforces the competitive dynamics identified by \citet{zinn1994market} and \citet{hackmann2017incentivizing}. Nursing homes are not just competing for residents; they are competing for a finite regional supply of labor. Our results suggest that this competition is increasingly mediated by third-party agencies, which act as "labor market brokers" that can either stabilize or destabilize entire geographic clusters. The "agency capture" we see in rural states suggests that \citet{gandhi2020private} and \citet{lord2023relationship}'s concerns about financial performance and competition are deeply tied to the spatial concentration of these labor costs. In a state like Vermont, where the mean ratio is 23\%, the agency has effectively become the primary employer of the nursing workforce, extracting a "rent" that drains resources from direct care.

\subsection{The Rural Paradox: Thin Markets and Agency Capture}

One of the most significant contributions of this study is the identification of rural hotspots in states like Vermont, North Dakota, and Montana. This finding presents a "rural paradox": while rural nursing homes often have lower resident turnover and stronger community ties, they are now demonstrating the highest levels of agency dependency. We hypothesize that this is a direct result of "thin labor markets." In urban areas, a facility that loses a nurse has a large pool of potential replacements. In a rural county, the loss of two or three permanent nurses can represent a significant percentage of the local labor supply.

When a staffing agency enters such a thin market, it can quickly achieve "agency capture." By offering slightly higher wages or more flexible schedules, the agency can recruit the few available nurses in the area. The local nursing homes, unable to compete with the agency's flexibility or wage offerings due to fixed Medicaid rates, are forced to hire those same nurses back through the agency at 1.5x or 2x the cost. This creates a "monopoly-like" power for the agency in the local market. Our change map (Figure 5) shows that this capture is not just persistent but is intensifying in these rural corridors, suggesting a permanent structural shift away from the traditional "community-based" staffing model.

\subsection{Theoretical Implications: Rethinking Organizational Boundaries}

Theoretically, our findings challenge the traditional view of the nursing home as a discrete organizational unit. In the context of "social contagion," the boundaries between facilities become porous. The staffing decisions made by Facility A directly constrain and shape the options available to Facility B. This suggests that "organizational structure" in long-term care should be reconceptualized as a networked property of a geographic cluster rather than an internal attribute of a single firm.

This shift has implications for how we apply Donabedian's SPO framework. If "structure" is external and regional, then "process" (the care provided) will also be regionally correlated. We may find that "best practices" or "poor practices" diffuse through these same labor networks as agency staff move between facilities. Future research should investigate whether the "contagion of labor" also leads to a "contagion of care quality," where clinical outcomes follow the same spatial patterns as agency reliance.

\subsection{Policy Implications: A Spatial Mandate for Stabilization}

The geographic clustering and contagion of contract staffing have profound implications for policy. Current regulatory approaches, such as the proposed federal minimum staffing mandates, largely treat nursing homes as independent actors. However, if a facility's staffing is heavily influenced by its neighbors, a facility-level mandate may be insufficient.

1. \textbf{Regional Workforce Stabilization:} Regulators should consider "cluster-based" interventions. If contract staffing is a regional phenomenon driven by contagion, then stabilizing a single facility without addressing its neighboring competitors may lead to "staffing leakage." State and federal agencies should monitor "hotspot" counties and provide targeted support to stabilize the permanent nursing pool across the entire local market.

2. \textbf{Rural Labor Pipeline Support:} The high ratios in rural states (VT, ND, MT) indicate that traditional market forces are failing to provide a stable nursing workforce. These areas require specific workforce pipelines, such as increased funding for local nursing schools, loan forgiveness programs for permanent rural nursing home staff, and perhaps distance-based reimbursement adjustments to offset the high cost of agency travel.

3. \textbf{Transparency and Margin Caps:} Given the clustering effect, state-level policies that cap staffing agency margins or increase price transparency may be effective. However, such policies must be coordinated regionally. If one state caps margins while its neighbor does not, agencies may simply shift their labor supply to the "uncapped" cluster, exacerbating the regional disparity.

\subsection{Limitations and Future Research}

This study is not without limitations that warrant acknowledgment. First, while PBJ data are highly accurate for hours worked, they do not include wage data or the specific reasons why a facility turned to contract labor. We are inferring "contagion" from staffing ratios, but we cannot definitively state whether the mechanism is price-driven competition, staff poaching, or a broader cultural shift in nursing work preferences. Future research should integrate facility-level financial reports and wage data to disentangle these economic drivers. Second, our spatial analysis at the county and state levels may mask finer-grained patterns. For instance, competition for labor in a high-density urban neighborhood (e.g., Manhattan) may follow different spatial rules than competition in a large, sparsely populated rural county in Montana.

Third, our proximity model is cross-sectional in its regression form. While we show trends in clustering over time using Moran's I, the OLS regression captures a "snapshot" of the contagion effect in 2024Q4. A truly dynamic spatial-temporal model, such as a spatial panel model with facility-level fixed effects, would provide deeper insights into the "velocity" of contagion and how shocks (like a new staffing mandate) propagate through the system over time. Fourth, we did not account for the role of nursing home chains or corporate ownership. It is highly possible that the observed contagion is partly "organizational" rather than purely spatial—if a large multi-state chain adopts a contract labor model, it may do so across all its facilities in a region simultaneously, creating a cluster that appears spatial but is actually driven by corporate strategy \citep{lu2011clustering, gandhi2020private}.

Furthermore, the data screening process, while necessary for statistical validity, may have excluded some facilities with genuine but extreme staffing challenges. While our strict outlier rule ($|z| > 4$) was designed to minimize bias, it is possible that we have slightly underestimated the intensity of the crisis in the most severely understaffed facilities. Future research should explore "tail-end" staffing behaviors using non-parametric or robust regression techniques that can accommodate extreme variation. Finally, we did not incorporate resident clinical outcomes into this study. While we have mapped the diffusion of the "structure" of staffing, the ultimate proof of the "epidemiological" model would be showing that clinical outcomes (the "disease") follow the same spatial-temporal diffusion patterns as the agency labor itself.

\section{Conclusion}

Contract staffing in U.S. nursing homes has undergone a fundamental transformation. What began as a nationwide crisis response has evolved into a structural feature of specific regional labor markets, sustained by geographic clustering and proximity-driven contagion. Our findings show that nursing homes are no longer making staffing decisions in isolation; they are part of a highly correlated spatial network where the practices of neighbors profoundly shape their own reliance on precarious labor. As the industry moves toward new federal staffing standards, policymakers must recognize that workforce stabilization is not just a facility-level challenge but a spatial epidemiological one. Only by addressing the regional market failures and contagion effects that drive agency dependency can we ensure a stable, permanent, and high-quality nursing home workforce for the future.

\newpage
\bibliographystyle{unsrtnat}
\bibliography{../01-literature/references}

\end{document}
